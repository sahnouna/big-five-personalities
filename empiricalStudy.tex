\section{Empirical Study/ Testing}



we tested our Data gender.R with P-value and T-student  to see whether it is significant or not  and then we compare  the mean of males and the mean of  females  to see which one of gender have a special personality 

we concluded that in this table :

\begin{table}[ht]
    \begin{center}
        {\footnotesize
        \begin{tabular}{l|l|l|l|l|l|}
        \hline \hline
               & introversion   & neuro    & agree   & openess   & conscient        \\
            \hline
                Mean of males    & -0,06 & -0,22   & -0,27 & -0,054 & 0,14   \\
                Mean of females  & 0,04 & 0,14 & 0,18 & 0,036 & -0,092  \\
                t-student        & 7 & 25 & 31,5 & 6 & 16   \\
                p-value          & 1,213e-12 & 2,2e-16 & 2,2e-16 & 3,373e-10 & 2,2e-16   \\
               
            \hline \hline
        \end{tabular}}
    \end{center}
    
\end{table}

{\normalsize{\bf the results:}} \\\vspace{0.5cm}


we see that the t-student is bigger than T-table wich equal 1,65  with α=0,05     and  Degree of freedom (DF) is bigger than 29
  P-value is smaller than 0,05
and this confirm our rejection hypothesis that mean not equal to zero (or the mean of the females not equal the mean of males) 
  in case Introversion/Extraversion , Neuro, Agree and Openess
we see that the mean of the males is smaller than  the mean of  the females and that mean females are more Introversion/Extraversion,neuro, agree and openness than males
But in case Consient we see that the mean of the males is bigger than  the mean of  the females and that mean males are more consient than females 
















