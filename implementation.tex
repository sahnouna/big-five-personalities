\section{Implementation/Simulation}


\underline{Data gender
\href{https://github.com/sahnouna/BIG_FIVE_PERSONALITIES/blob/master/DataAnalysis.R}{\includegraphics[scale = 0.06]{Figures/qletlogo.pdf}} :} 
\newline 
This Quantlet translates data sets containing answers to a 50 item questionnaire into a data frame with the values of the personality traits according to the evaluation key. Optionally it can also deal with a data set from a questionnaire about grit. However, the data needs to contain the 50 items of the personality test. The order in which these answers are given does not matter since they are going to be reordered in the process of data cleaning.
\newline
This file deals with an analysis of all five personality traits (Extraversion, Openness, Conscientiousness, Agreeableness, and Neuroticism) with accordance to the gender(males and females) . In short, we analysis the personality among to compare the mean of males and the mean of females . Note, that the analysis from this file was made for a visual interpretation.
 \newline \newline
 In our study we have tried to show such patterns as well. For that we wanted to receive a visual big picture of the personality traits across the gender with accordance to our data set. 
Our file, in which we applied this study, starts with the general data generation part. With the function source(), we are able to access another file or rather script in order to connect with the already prepared data set which we will need for this study
The data "clean" function 
The data "clean" function mean takes the name of a data set, a boolean variable whether the data contains grit questionnaire and a survey date as parameters. This function can deal with excel data and data in the .csv format. It converts some of the numerical variable into factors and reorders the column. Before columns are converted the function checks whether they are null to prevent errors when trying to convert non existing columns.
\newline
\underline{DataAnalysis
\href{https://github.com/sahnouna/BIG_FIVE_PERSONALITIES/blob/master/DataAnalysis.R}{\includegraphics[scale = 0.06]{Figures/qletlogo.pdf}} :} 
\newline 

This Quantlet have This file deals with an analysis of all five personality traits (Extraversion, Openness, Conscientiousness, Agreeableness, and Neuroticism) with accordance to the gender(males and females) in two diffrents  country(USA and GB) . In short, we analysis the personality among to compare the mean of males and the mean of females . Note, that the analysis from this file was made for a visual interpretation
 In our study we have tried to show such patterns as well. For that we wanted to receive a visual big picture of the personality traits across the gender with accordance to our data set. 
Our file, in which we applied this study, starts with the general data generation part. With the function source(), we are able to access another file or rather script in order to connect with the already prepared data set which we will need for this study
The data "clean" function 
The data "clean" function mean takes the name of a data set, a boolean variable whether the data contains grit questionnaire and a survey date as parameters. This function can deal with excel data and data in the .csv format. It converts some of the numerical variable into factors and reorders the column. Before columns are converted the function checks whether they are null to prevent errors when trying to convert non existing columns.
\newline
\underline{Data Preparation  \href{https://github.com/sahnouna/BIG_FIVE_PERSONALITIES/blob/master/DataPreparation.R}{\includegraphics[scale = 0.06]{Figures/qletlogo.pdf}} :} 

In this Quantlet we test how well principal component analysis and factor analysis estimate the scaled values obtained by using the evaluation key. Throughout this analysis the data worked with will often be just the answers to the questionnaires. To that end the columns will be selected using to numbers. 'Start' which is set to "which(colnames(data) == "E1")" and 'end' which is set to 'start + 49'. This is possible since the columns where sorted in this specific way during data cleaning.
\newline
The first step is estimating the correct number of pcs or factors to extract. This is done in several ways.
The first way is using the function implemented in R, which estimates the number of factors in a factor analysis. Another method is to do a parallel analysis to find both a good number for factors and pcs. The last method implemented is to do a simple PCA and to draw a screeplot. 
\newline
In the code there are 3 functions implemented to extract the values via PCA as well as 1 function that uses factor analysis . Those 4 functions all take a cleaned data frame as input and return a new data frame in which the questionnaire answers are replaced by the estimated values for the personality traits. Non-questionnaire columns remain unchanged. The functions for PCA differ in the PCA function they use.  
\newline
Next the results of 'prcomp' and 'princomp' are compared in 2 ways. First, a screeplot is drawn for both results and then their centers and standard deviations are compared.
\newline
The main comparison of PCA, factor analysis and the evaluation key is done in two functions: 'compareDensities' and 'compareDifferences' . Both function take the name of a data file as input. Then both transform the questionnaire answers to personality traits using the evaluation key, factor analysis and PCA ('princomp'). 'compareDensities' then uses a for loop to go through a vector with the names of the personality traits and draws a a plot containing the density distribution of all 3 methods for each personality trait.
\newline
'compareDifferences' simply compares the average difference between factor analysis and evaluation key vs. PCA and evaluation key.
\newline
In the data cleaning part it was mentioned that there is an option to combine 2 data set in order to have more observations. This is only possible if both data sets don't have overlap. Otherwise you would have the several observations twice in the combined data set. 
\newline
The last part of this analysis is a test to check for overlap between 2 datasets. To do this the 2 data sets are merged by the traits or by the remaining columns. The 'nrow' function then gives the number of matching pairs in the 2 data sets.  

