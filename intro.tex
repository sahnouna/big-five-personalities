\section{Introduction}

The Big Five Personality Traits model can help to address these issues. It's a test that can be used to measure a person's most important personality characteristics, and help him to understand which roles suit him best. Recruiters can also use it to find people who have the personality, as well as the skills, to fit the roles that they are hiring for.

Five-dimension personality model presented by Goldberg
is termed as big five model in personality research
(Goldberg, 1992; Sucier and Goldberg, 1998).

The big five personality dimensions include firstly, opennessess
to experience which is the inclination to be imaginative,
independent, and interested in variety. Secondly,
Conscientiousness is the affinity to be prepared, chary,
and disciplined. Thirdly, the propensity to be gregarious,
fun-loving, and warm is known as extraversion. Fourthly,
the tendency to be sympathetic, trusting, and supportive
is termed as Agreeableness. Lastly, neuroticism is the
tendency to be anxious, emotionally destable, and selfblaming (Goldberg, 1993). Research suggests that there is a significant relationship between personality type and career choices but in practice wrong career choice are made due to the ignorance of specific personality type of the individuals
When working with a large data set, it can be useful to represent the entire data set with Numerical methods. Our project consists to find out what could be the reason for someone to have a certain behavior  at the work place. So we worked on the big five personalities :  Introverted/Extroverted,  Neuro , Agreer, Opened  and Conscious .


Many people, perceive the differences between men and women to be large – so large, in fact, that communication between genders may be difficult. Countless examples from popular culture reinforce this view of extreme differences between the sexes – but is it accurate? Men and women have obviously different biological roles when it comes to propagation of the species, but how much they differ psychologically is a more controversial question, one that requires empirical research to answer adequately. Whether the underlying causes of psychological gender differences are evolutionary or socio-cultural, understanding how men and women differ in the ways in which they think, feel, and behave can shed light on the human condition

i specify my work about gender and we are trying to find evidence of this statement by operating some statistical methods with the Statistical Program " R in data analysis and We went to the Hypothesis women were more opened and conscious  than men at the work place
 
 
 

   

\section{Theory and Design}
how to compare two density distributions in males and females according to the Big Five Personalities
